\documentclass[pdf,xcolor={svgnames}]{beamer}

\usetheme{frama-c}

\usepackage{tikz}
\usepackage{graphbox}
\usepackage[overlay,absolute]{textpos}
\usepackage[type={CC},modifier={by}]{doclicense}

\usetikzlibrary{mindmap}
\hypersetup{colorlinks=true,urlcolor=blue}

\author{Virgile Prevosto\\
\href{mailto:virgile.prevosto@cea.fr}{virgile.prevosto@cea.fr}}
\institute{CEA Tech List}
\title{C Program Specification and Verification\\ with ACSL and Frama-C/WP}
\subtitle{VerifyThis Tutorial}
\date{2019-04-06}

\begin{document}

\begin{frame}\maketitle
\begin{textblock*}{1.5cm}(6mm,87mm)
\doclicenseImage[imagewidth=8mm]
\end{textblock*}
\end{frame}

\begin{frame}{Overview of the tutorial}
\begin{itemize}
\item short general introduction to Frama-C and ACSL
\item examples of writing ACSL specifications
\item verification of implementations with the WP plugin of Frama-C
\item All material available on \href{https://github.com/Frama-C/open-source-case-studies/tree/master/tutorials/2019-04-VerifyThis}{Frama-C github}:
\begin{center}
\url{https://frama.link/TODO}\hspace{5mm}
\includegraphics[width=1cm,align=c]{tuto-qr.pdf}
\end{center}
\begin{itemize}
\item \alert{\texttt{Examples}} contains plain (unannotated) C code
\item \alert{\texttt{Solutions}} contains the corresponding annotations...
\item ... that should be provable by Frama-C 18.0, Alt-Ergo and Coq
\end{itemize}
\end{itemize}
\end{frame}

\begin{frame}{It's 2019! Why bother with proving C programs?}
\begin{itemize}
\item Lot of legacy code
\item Embedded world (aka IoT) still uses it in many places
\item And in some cases they care about safety and (cyber)security
\end{itemize}
\begin{block}{A few recent use cases}
\begin{itemize}
\item \href{https://gitlab.com/systerel/S2OPC/}{S2OPC}
  OPC (communication protocol for industrial systems), result of INGOPCS
  French project
\item \href{https://www.bureauveritas.com/white-papers/cybersecurity-guidelines-for-development-and-assessment-bv-sw-200}{Bureau Veritas Cybersecurity Guidelines}
\item \href{https://www.vessedia.eu/}{Vessedia H2020 project} \includegraphics[align=c,width=1.5cm]{vessedia-logo}\\
including verification of parts of \href{http://www.contiki-os.org/}{Contiki OS}
\end{itemize}
\end{block}
\end{frame}

\begin{frame}{Frama-C at a glance}
\begin{itemize}
\item
  A \alert{Fra}mework for \alert{m}odular \alert{a}nalysis of \alert{C} code.
\item \url{http://frama-c.com/}
\item Developed at CEA Tech List and Inria
\item Released under LGPL license (v18.0 Argon in November 2018)
\item Kernel based on CIL (Necula et al. -- Berkeley).
\item ACSL annotation language.
\item Extensible platform
\begin{itemize}
\item Collaboration of analyses over same code
\item Inter plug-in communication through ACSL formulas.
\item Adding specialized plug-in is easy
\end{itemize}
\end{itemize}
\end{frame}

\begin{frame}[fragile]{Frama-C plugins}
\tikzset{every node/.append style={concept,font=\Tiny,text width=8mm,minimum size=0mm,inner sep=0pt}}
\begin{center}
\begin{tikzpicture}[
  small mindmap,
  concept color=OrangeRed,
  level 1 concept/.append style={level distance=18mm},
  level 2 concept/.append style={level distance=14mm}
]

\node{Frama-C\\ Plug-Ins}
  [counterclockwise from=30]
  child[concept color=Orange] {
    node { Specification Generation }
    [counterclockwise from=-60]
    child { node{Aoraï} }
    child { node{CaFE} }
    child { node{RPP} }
    child { node{MetACSL} }
  }
  child[concept color=Gold] {
    node {Verification}
    [counterclockwise from = 80]
    child { node{Eva} }
    child { node{WP} }
  }
  child[concept color=Coral] {
    node {Code Navigation}
    [counterclockwise from = 90]
    child { node{Studia} }
    child { node{Impact} }
    child { node{Slicing} }
  }
  child[concept color=LightSalmon] {
    node { Dynamic Analysis }
    [counterclockwise from =160]
    child { node{E-ACSL} }
    child { node{PathCrawler} }
    child { node{LTests} }
  }
  child[concept color=Orange!50!white] {
    node { others }
    [counterclockwise from=220]
    child { node{Mthread} }
    child { node{Pilat} }
    child { node{JCard} }
  }
;
\end{tikzpicture}
\end{center}
\end{frame}

\end{document}
