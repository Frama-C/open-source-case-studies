\documentclass[pdf,xcolor={svgnames}]{beamer}

\usetheme{frama-c}

\usepackage{tikz}
\usepackage{graphbox}
\usepackage[overlay,absolute]{textpos}
\usepackage[type={CC},modifier={by}]{doclicense}
\usepackage{listings}
\usepackage{pifont}
\usepackage{courier}

\usetikzlibrary{mindmap}
\usetikzlibrary{decorations.text}
\usetikzlibrary{decorations.pathreplacing}

\hypersetup{colorlinks=true,urlcolor=blue}

\author{Virgile Prevosto\\
\href{mailto:virgile.prevosto@cea.fr}{virgile.prevosto@cea.fr}}
\institute{CEA Tech List}
\title{C Program Specification and Verification\\ with ACSL and Frama-C/WP}
\subtitle{VerifyThis Tutorial}
\date{2019-04-06}

\definecolor{darkgreen}{rgb}{0, 0.5, 0}
\definecolor{lightgreen}{rgb}{0.9,1,0.9}
\definecolor{lightred}{rgb}{1,0.7,0.7}
\definecolor{lightblue}{rgb}{0.8,0.8,1}
\definecolor{darkblue}{rgb}{0.3,0.3,1}
\definecolor{lightyellow}{rgb}{1,1,0.5}
\definecolor{darkyellow}{rgb}{0.8,0.8,0}
\definecolor{darkred}{rgb}{0.33,0.14,0.09}

\colorlet{perimetercolorbg}{lightblue}
\colorlet{perimetercolorfg}{darkblue}
\colorlet{propertycolorbg}{lightgreen}
\colorlet{propertycolorfg}{darkgreen}
\colorlet{tcbcolorbg}{lightyellow}
\colorlet{tcbcolorfg}{darkyellow}

\def\good{\textcolor{darkgreen}{\ding{52}}}
\def\bad{\textcolor{red}{\ding{56}}}

\lstdefinelanguage{ACSL}{%
  morekeywords={assert,assigns,assumes,axiom,axiomatic,behavior,behaviors,
    boolean,breaks,complete,continues,data,decreases,disjoint,ensures,
    exit_behavior,ghost,global,inductive,integer,invariant,lemma,logic,loop,
    model,predicate,reads,real,requires,returns,sizeof,strong,struct,terminates,type,
    union,variant,volatile,writes},
%  otherkeywords={\\at,\\base_addr,\\block_length,\\false,\\fresh,\\from,
%                 \\initialized,\\lambda,\\let,\\match,\\max,\\nothing,\\null,
%                 \\numof,\\old,\\result,\\specified,\\strlen,\\sum,\\true,
%                 \\valid,\\valid_range},
  keywordsprefix={\\},
  alsoletter={\\},
  morecomment=[l]{//},
  moredelim=*[l]{//@},
  moredelim=*[s]{/*@}{*/}
}

\lstset{
basicstyle=\ttfamily,
keywordstyle=\bfseries\color{red},
showstringspaces=false,
stringstyle={\rmfamily\color{gray}},
escapechar=^
}

\begin{document}

\begin{frame}\maketitle
\begin{textblock*}{1.5cm}(6mm,87mm)
\doclicenseImage[imagewidth=8mm]
\end{textblock*}
\end{frame}

\begin{frame}{Overview of the tutorial}
\begin{itemize}
\item short general introduction to Frama-C and ACSL
\item examples of writing ACSL specifications
\item verification of implementations with the WP plugin of Frama-C
\item All material available on \href{https://github.com/Frama-C/open-source-case-studies/tree/master/tutorials/2019-04-VerifyThis}{Frama-C github}:
\begin{center}
\url{https://frama.link/fc-tuto-2019-04}\hspace{5mm}
\includegraphics[width=1.7cm,align=c]{tuto-qr.pdf}
\end{center}
\begin{itemize}
\item \alert{\texttt{Examples}} contains plain (unannotated) C code
\item \alert{\texttt{Solutions}} contains the corresponding annotations...
\item ... that should be provable by Frama-C 18.0, Alt-Ergo and Coq
\end{itemize}
\item \texttt{vagrant init virgile/VerifyThis-fc-tutorial} (very experimental)
\end{itemize}
\end{frame}

\begin{frame}{It's 2019! Why bother with proving C programs?}
\begin{itemize}
\item[\bad] Lack of modularity and high-level constructs
\item[\bad] Many quirks in the semantics (pointers)
\item[\bad] Small standard library
\item[\good] Lot of legacy code
\item[\good] Embedded world (aka IoT) still uses it in many places
\item[\good] And in some cases they care about safety and (cyber)security
\end{itemize}
\end{frame}

\begin{frame}{A few recent use cases}
\begin{itemize}
\item \href{https://gitlab.com/systerel/S2OPC/}{S2OPC}
  OPC (communication protocol for industrial systems), result of INGOPCS
  French project
\item \href{https://www.bureauveritas.com/white-papers/cybersecurity-guidelines-for-development-and-assessment-bv-sw-200}{Bureau Veritas Cybersecurity Guidelines}
\item \href{https://www.vessedia.eu/}{Vessedia H2020 project} \includegraphics[align=c,width=1.5cm]{vessedia-logo}\\
including verification of parts of \href{http://www.contiki-os.org/}{Contiki OS}
\end{itemize}
\end{frame}

\begin{frame}{Frama-C at a glance}
\begin{itemize}
\item
  A \alert{Fra}mework for \alert{m}odular \alert{a}nalysis of \alert{C} code.
\item \url{http://frama-c.com/}
\item Developed at CEA Tech List and Inria
\item Released under LGPL license (v18.0 Argon in November 2018)
\item Kernel based on CIL (Necula et al. -- Berkeley).
\item ACSL annotation language.
\item Extensible platform
\begin{itemize}
\item Collaboration of analyses over same code
\item Inter plug-in communication through ACSL formulas.
\item Adding specialized plug-in is easy
\end{itemize}
\end{itemize}
\end{frame}

\begin{frame}[fragile]{Frama-C plugins}
\tikzset{every node/.append style={concept,font=\Tiny,text width=8mm,minimum size=0mm,inner sep=0pt}}
\begin{center}
\begin{tikzpicture}[
  small mindmap,
  concept color=OrangeRed,
  level 1 concept/.append style={level distance=18mm},
  level 2 concept/.append style={level distance=14mm}
]

\node{Frama-C\\ Plug-Ins}
  [counterclockwise from=30]
  child[concept color=Orange] {
    node { Specification Generation }
    [counterclockwise from=-50]
    child { node{Aoraï} }
    child { node{CaFE} }
    child { node{RPP} }
    child { node{MetACSL} }
  }
  child[concept color=Gold] {
    node {Verification}
    [counterclockwise from = 80]
    child { node{Eva} }
    child { node{WP} }
  }
  child[concept color=Coral] {
    node {Code Navigation}
    [counterclockwise from = 90]
    child { node{Studia} }
    child { node{Impact} }
    child { node{Slicing} }
  }
  child[concept color=LightSalmon] {
    node { Dynamic Analysis }
    [counterclockwise from =160]
    child { node{E-ACSL} }
    child { node{PathCrawler} }
    child { node{LTests} }
  }
  child[concept color=Orange!50!white] {
    node { others }
    [counterclockwise from=230]
    child { node{Mthread} }
    child { node{Pilat} }
    child { node{Frama-Clang} }
  }
;
\end{tikzpicture}
\end{center}
\end{frame}

\begin{frame}{What is verified by Frama-C?}
\begin{minipage}[t]{0.32\textwidth}
\begin{block}{Code Properties}
\fcolorbox{propertycolorfg}{propertycolorbg}{
\parbox{0.9\textwidth}{
\begin{itemize}
\item Functional properties (contract)
\item Absence of run-time error
\item Termination
\item \textcolor{gray}{Dependencies}
\item \textcolor{gray}{Non-interference}
\item \textcolor{gray}{Temporal properties}
\end{itemize}
}}
\end{block}
\end{minipage}
\hfill
\begin{minipage}[t]{0.6\textwidth}
\begin{block}{\mbox{Perimeter of the verification}}
\fcolorbox{perimetercolorfg}{perimetercolorbg}{
\parbox{0.9\textwidth}{
\begin{itemize}
\item Which part of the code is under analysis?
\item Which initial context?
\end{itemize}
}}
\end{block}
\vskip0.7cm
\begin{block}{\mbox{Trusted Code Base}}
\fcolorbox{tcbcolorfg}{tcbcolorbg}{
\parbox{0.9\textwidth}{
\begin{itemize}
\item ACSL Axioms
\item Hypotheses made by analyzers
  \begin{itemize}
    \item \alert{WP memory model}
  \end{itemize}
\item Stub Functions
\item Frama-C itself
\end{itemize}
}}
\end{block}
\end{minipage}
\end{frame}

\begin{frame}[fragile,label=ACSL]{ANSI/ISO C Specification Language}
\begin{block}{Presentation}
\begin{itemize}
\item Based on the notion of contract, like in Eiffel
\item Allows users to specify functional properties of their code
\item Allows communication between various plugins
\item Independent from a particular analysis
\item \url{https://github.com/acsl-language/acsl}
\end{itemize}
\end{block}
\begin{block}{Basic Components}
\begin{itemize}
\item First-order logic
\item Pure C expressions
\item C types + $\mathbb{Z}$ (integer) and $\mathbb{R}$ (real)
\item Built-ins predicates and logic functions, particularly over
pointers.
\end{itemize}
\end{block}
\end{frame}

\begin{frame}[fragile]{Integer Arithmetic in ACSL}
\lstset{language=ACSL}
\begin{itemize}
\item All operations are done over $\mathbb{Z}$: \alert{no overflow}
\item ACSL predicate \lstinline|INT_MIN <= x + y <= INT_MAX|\\
 $\Leftrightarrow$\\
 C operation \lstinline|x+y| does not overflow (undefined behavior)
\item $\text{\lstinline|(int)z|}\equiv\text{\lstinline|z|}\mod 2^{8*\text{\lstinline|sizeof(int)|}}$
\item and \lstinline|INT_MIN <= (int)z <= INT_MAX|
\end{itemize}
\end{frame}

\begin{frame}[fragile]{Memory description in ACSL}
\lstset{language=C}
\begin{tikzpicture}[x=6mm,y=6mm]
\fill[lightred] (0,0) rectangle (2,1);
\fill[LightGrey] (2,0) rectangle (4,1);
\fill[lightgreen] (4,0) rectangle (8,1);
\draw[decoration=brace,decorate] (0,1.1) -- (2,1.1) node[midway,above]{\tiny\lstinline|s[0].x|};
\draw[decoration=brace,decorate] (2,1.1) -- (4,1.1) node[midway,above]{\tiny padding};
\draw[decoration=brace,decorate] (4,1.1) -- (8,1.1) node[midway,above]{\tiny \lstinline|s[0].y|};
\fill[lightred] (8,0) rectangle (10,1);
\fill[LightGrey] (10,0) rectangle (12,1);
\fill[lightgreen] (12,0) rectangle (16,1);
\draw[decoration=brace,decorate] (8,1.1) -- (10,1.1) node[midway,above]{\tiny\lstinline|s[1].x|};
\draw[decoration=brace,decorate] (10,1.1) -- (12,1.1) node[midway,above]{\tiny padding};
\draw[decoration=brace,decorate] (12,1.1) -- (16,1.1) node[midway,above]{\tiny \lstinline|s[1].y|};
\draw[step=1] (0,0) grid (16,1);
\node (fst) at (0,-1) {\lstinline|&s[0]|};
\draw[->] (fst) -- (0,0);
\node (snd) at (8,-1) {\lstinline|&s[1]|};
\draw[->] (snd) -- (8,0);

\end{tikzpicture}
\begin{minipage}{0.3\textwidth}
\begin{lstlisting}
struct S {
short x;
int y;
} s[2];
\end{lstlisting}
\end{minipage}
\begin{minipage}{0.65\textwidth}
\lstset{language=ACSL}
\begin{lstlisting}
\valid(&s[0]+(0 .. 1))
\valid((char*)&s[0] + (0 .. 15))
!\initialized(*((char*)&s[0].x+2))
\block_length(&s[0]) == 16
\base_addr(&s[0].y) == s
\offset(&s[1].y) == 12
\separated(&s[0],&s[1])
\end{lstlisting}
\end{minipage}
\end{frame}

\end{document}
